\section{Step 1.简要AGI综合行研}
*注:因为大模型是迈向AGI目前为止可行性最高的实现路径,因此以下分析和研究均围绕大模型或者基于大模型的多模态模型展开。
\subsection{大模型发展标志性事件梳理}
\begin{enumerate}
    \item 2015年,OpenAI成立。
    \item 2017年,Google推出Transformer模型,参数量首次达到上亿规模。
    \item 2018年,Google推出Bert模型,参数量达到3亿规模;OpenAI推出GPT。
    \item 2019年,OpenAI推出GPT-2模型,参数量达到15亿规模。
    \item 2020年,OpenAI推出GPT-3模型,参数量达到1750亿;微软和英伟达发布MT-NLG模型,参数量达到5300亿。
    \item 2021年,Google发布Switch Transformer模型,参数量达到1.6万亿。
    \item 2022年,OpenAI推出基于GPT-3.5的ChatGPT,成为现象级产品。
    \item 2023年,OpenAI推出GPT-4,参数规模和细节尚未公开。
\end{enumerate}
大(语言)模型作为人类在人工智能领域最先进的产出,同时也是最接近实现AGI的产品,其发展极其迅速。并且有进一步加速的趋势。
\subsection{大模型时代下的核心竞争力}
\begin{enumerate}
    \item 算法
    \item 数据
    \item 算力
\end{enumerate}
算法或者模型本身,本质是其相关的研发实力,(同时也是在AGI领域科技树的某个技术分支的押注)是一个公司的最强的壁垒和发展护城河。越先进的模型,越可能打造出优质的产品,才能吸引到广大消费者群体和后续的商业化,因此算法是做AGI公司的核心竞争力。

数据是训练的基础,训练数据质量的好坏、范围直接影响模型训练的结果,从这点来看,模型决定了效果的上限,但是数据决定了效果的实际完成度。当数据和模型都配套好才可以发挥模型的最好的效果。

算力是发展大模型的前提。在当前的模型参数下,没有庞大算力的公司意味着可能直接错失进入AGI行业的入场券,这也意味着当前AGI行业只能是算力充足、资金雄厚、研发实力强劲的巨头可以参与。同时因为NVIDA的A100销售禁令,可能会造成国内算力缺口。

\subsection{中国和北美大模型竞争格局}
首先需要理清一个逻辑是:不是领先的大模型占据高端市场,落后一些的大模型占据低端市场,而是领先的大模型几乎会占领一切市场。和传统的高端——中端——低端市场有与之相对应的高端——中端——低端消费群体不同,民用大模型的使用成本已经降低到一个较低的水平(按照GPT-4的订阅费用来衡量),因此高端中端低端用户都可以消费起行业最领先的模型和产品,同时因为行业头部的模型(GPT-4)的使用体验要远大于中部的模型(百度的文心一言、阿里的通义千问等国产大模型),会导致各个市场的用户会集中使用最好的模型(除非有不可抗因素,如IP禁用、大陆地区封号等等)。

因此这对于中国发展大模型是有严重劣势的,因为国产大模型可能和北美有1-2代的版本代差,而这个追赶期目前仍不确定。\textit{“李彦宏称自己前段时间接受采访时说跟ChatGPT的差距大约是两个月,有点断章取义,因为自己后面紧接着说:‘这不是重点,重点是这两个月的差距我们要用多长时间才能赶上,也许很快,也许永远也赶不上。’”}

我们从“大模型时代下的核心竞争力”梳理的三个角度来进行分析。

\noindent\textbf{模型:}
\begin{itemize}
    \item 北美有顶流——OpenAI的GPT-4及GPT-3.5,号称ChatGPT的最强竞品——anthropic的Claude,传统AI巨头Meta的LLaMA,Amazon的Amazon Titan等等。
    \item 国内有百度的文心一言、阿里的通义千问、华为的盘古、科大讯飞的星火认知大模型、商汤的SenseChat、360智脑、腾讯的混元等。
\end{itemize}
由于大部分均仍在内测阶段,无法对其进行完整的测试。但是根据一份来源于真格基金的测评来看,国内最强的模型是百度的通义千问,综合表现较好的商汤的SenseChat。但是仍然不及ChatGPT-3.5。从直接的测试结果来看,国内最好的模型效果是ChatGPT-3.5的75\%左右。


\noindent\textbf{数据:}
从中文语料库和英文语料库来看,都应该是足够训练大模型的,但是国内的数据标注产业可能相较于北美较为落后。

\noindent\textbf{算力:}
算力问题可能是国内最明显的短板,最主要的原因就在于Nvida的A100禁令,国内有超大规模算力(1万块A100规模)的公司屈指可数。

\subsection{洞见}
\begin{enumerate}
    \item 整体来看,大模型产业北美从各个方面都处于领先地位,国产大模型在追赶,但是处于一个十分危险的境地(好在有Great firewall)。这次的AGI竞赛可能会直接关系到一些互联网科技公司的生死存亡。
    \item 国内有可能的发展是数据标注行业,虽然相对北美较为弱势,但是行业门槛较低,如果发挥一定人力成本优势,或许存在发展空间。
    \item 比较看好的是百度和华为的模型,因为具有先发优势,同时两家都在AI领域也有积累。其次是字节的模型,字节在AI领域的积累也比较深厚,但是由于发展时间较晚,可能还存在一定的落地期。
    \item 在这种情况下,大厂可能需要两条腿走路——既要发展自己的大模型,又要积极支持和收购在技术路线上有正确探索的中小公司。
    \item 在GPT部署使用场景下,云服务和向量数据库可能会更大规模使用,需要关注有相关业务的公司。
\end{enumerate}

