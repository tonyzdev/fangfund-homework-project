\section{Step 2 模拟决策}
\subsection{1000万人民币筹码的投资策略}
1000万的资金按照2:8的二级市场投资和一级市场投资划分。
个人的建议如下:
\begin{enumerate}
    \item 200万中按照4:2:2:1:1的比例分别投资Nvida、Microsoft、Google、Amazon、百度股票。
    \item 800万中投资3-4家做\textbf{数据标注}的Startup的天使轮。*如果有可能,投资1-2个\textbf{非常规半导体算力}的Startup天使轮。或者一些做\textbf{向量数据库}的Startup天使轮。前提是技术路线可落地、前景好、靠谱等。
\end{enumerate}

\subsection{1亿元人民币筹码的投资策略}
相较于1000万筹码,1亿元可以到达给几个小型公司A——B轮融资的程度。
但是个人认为,和1000万人民币级别的投资策略不会有太大区别。因为可投资的范围(一级市场)到不了做算力、模型的公司上。
可能到20亿人民币规模会存在一些明显的改善,因为当前做的一些比较优秀的公司估值普遍在20亿美元以上。\textit{2023年3月9日Anthropic筹资11亿美元,投资者包括谷歌、Skype联合创始人Jaan Tallinn和FTX前首席执行官Sam Bankman-Fried等。}
因此对于1亿人民币规模的筹码而言,个人的建议如下:
\begin{enumerate}
    \item 20\%资金按照4:2:2:1:1的比例分别投资Nvida、Microsoft、Google、Amazon、百度股票。
    \item 80\%资金投资3-4家做\textbf{数据标注}的Startup的A轮。*如果有可能,投资1-2个\textbf{非常规半导体算力}的Startup的A轮。或者1-2个做\textbf{向量数据库}的Startup的A轮。前提是技术已经有一定落地,发展成本可控、前景较为明确,靠谱等。
\end{enumerate}













